\section{Bond graph modeling of dynamical systems}
Bond graphs describe systems as the interconnection of subsystems, or elements.
Each element has a $port$ through which it can be connected to its environment.
The ports of elements are interconnected by bonds that represent the bi-directional flow of two power variables, effort and flow.
The product of the effort and flow determines the rate of energy exchange between elements per unit of time, i.e. the power.
Power and energy are domain neutral phenomena.
Effort and flow are the generalized description of the signals that make up power in specific engineering domains.
For example, in mechanical systems power is the product of force (effort) and velocity (flow) and in electrical systems power is the product of voltage (effort) and current (flow).
For this reason, force and velocity are said to be the \textit{analog} of voltage and current, respectively.\footnote{In alternative, equally valid analogy voltage is the analog of velocity.\cite{}.}

The basic bond graph elements are divided into 1-port, 2-port, and multi-port elements.
The 1-port elements are the inertia, the capacitor, the resistor, and the active effort source and flow source.
Each 1-port element has a constitutive relation that relates the two power variables.
In case of the inertia and the capacitor this relation includes a time integral that keeps track of a stored state variable. 
The constitutive relation of a inertia is $f= \phi_I^{-1}(p)$, where $p=\int e dt$ and a capacitor is $e = \phi_C (q),$ where $q =\int f dt$,
The dynamics of the state variables are defined by the entire bond graph model.
This time integral also implies a preferred causality, i.e. a preferred effort or flow input.
The resistor dissipates energy and does not have a preferred causality; its causality depends on how it is interconnected to the rest of the system.
A resistor may return an effort for a flow input, $e=\phi_R(f)$, or return a flow for an effort input, $e=\phi_R^{-1}(e)$.
The 2-port elements are the transformer and gyrator.
A transformer links one effort $e_1$ to another effort $e_2$ and one flow $f_1$ to another flow $f_2$ in a power-preserving manner, i.e. $e_1f_1=e_2f_2$.
The gyrator links one effort $e_1$ to a flow $f_2$ and vice versa, again in a power-preserving manner.
The multi-port elements are the 0-junction and the 1-junction.
The multi-port elements represent the interconnection structure between elements. 
The 0-junction is an interconnection structure in which the flow variables sum up to zero and the effort variable is common,$f_1+\dots+f_n=0$, $e_1=\dots =e_n$ , e.g. an electrical node.
In the interconnection structure of the 1-junction the effort variables sum up to zero and the flow variable is common, $e_1+\dots+e_n=0$, $f_1=\dots =f_n$, e.g. an electrical loop.

\begin{tcolorbox}
make tables.
\end{tcolorbox}

\subsection{Bond graphs for economics}
Brewer extended the bond graph analogy in \cite{} by choosing the rate of orders as flow variable and the unit commodity price as effort variable.
The motivation behind this choice is that the power conservation at junctions,
\begin{equation*}
    e_1f_1+e_2f_2+\dots e_nf_n=0,
\end{equation*}
represents Walras' law for proper cash accounting.

However, choosing price as an effort variable entails that the time-integral of price, which Brewer calls the \textit{economic impulse}, is the economic generalized momentum.




This choice of variables, in particular the choice of price as an effort variable excludes the possibility of deriving price dynamics.

We propose to view price not as an effort variable, but as a generalized momentum.
This implies that the effort becomes a signal that provokes a rate of price movement, analogous to how a force provokes a rate of change in momentum.
In general, we refer to the economic effort variable as a \textit{desirability}.

Using this analogy, the power conservation at junctions represents the conservation of growth in an economic system. 
